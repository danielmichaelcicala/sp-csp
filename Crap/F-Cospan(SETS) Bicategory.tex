\documentclass[10pt,a4paper]{article}
\usepackage[latin1]{inputenc}

%--------------- LOAD PACKAGES ---------------------------------

\RequirePackage{amsfonts}
\RequirePackage{amsthm}
\RequirePackage{amssymb}
\RequirePackage{comment}
\RequirePackage{stmaryrd}
\RequirePackage{etoolbox}
\RequirePackage{mathtools}
\RequirePackage{amsmath}
\RequirePackage{mdwlist}
\RequirePackage{mathrsfs}
\RequirePackage{enumitem}
\setlist{itemsep=0em, topsep=0em, parsep=0em}
\setlist[enumerate]{label=(\alph*)}
\RequirePackage{tikz, tikz-cd}
\usetikzlibrary{matrix,arrows}
\RequirePackage{multicol}
\RequirePackage{arydshln,leftidx}
\RequirePackage[breaklinks,hidelinks]{hyperref}
\hypersetup{colorlinks,linkcolor={blue}}
\RequirePackage{subfiles}
\RequirePackage{makeidx}
\RequirePackage{multirow}
\RequirePackage{graphicx}

%------------------------ NEW COMMANDS --------------------------

\newcommand{\RR}{\mathbb{R}}
\newcommand{\ZZ}{\mathbb{Z}}
\newcommand{\NN}{\mathbb{N}}
\newcommand{\QQ}{\mathbb{Q}}
\newcommand{\CC}{\mathbb{C}}
\renewcommand{\epsilon}{\varepsilon}

\newcommand{\entry}[1]{\section*{#1} \index{#1} \label{#1}}
\newcommand{\cat}[1]{\mathtt{#1}}
\newcommand{\define}[1]{\emph{#1}}
\newcommand{\cl}[1]{\mathcal{#1}}
\newcommand{\scr}[1]{\mathscr{#1}}
\newcommand{\op}[1]{\operatorname{#1}}
\renewcommand{\t}[1]{\textup{#1}}
\renewcommand{\and}{\textup{ and }}

\newcommand{\from}{\colon}
\newcommand{\xto}[1]{\xrightarrow{#1}}
\newcommand{\unto}{\leftarrow}
\newcommand{\xunto}[1]{\xleftarrow{#1}}
\newcommand{\sm}{\smallsetminus}
\renewcommand{\(}{\left(}
\renewcommand{\)}{\right)}
\renewcommand{\ss}[2]{_{#1}^{#2}}
\renewcommand{\{}{\left\lbrace}
\renewcommand{\}}{\right\rbrace}
\renewcommand{\hat}{\widehat}
\renewcommand{\tilde}{\widetilde}
\renewcommand{\bar}{\overline}


%-------------- DECLARE MATH OPERATORS --------------------

\DeclareMathOperator{\Hom}{Hom}
\DeclareMathOperator{\id}{id}
\DeclareMathOperator{\ob}{Ob}
\DeclareMathOperator{\arr}{arr}
\DeclareMathOperator{\im}{im}
\DeclareMathOperator{\Aut}{Aut}
\DeclareMathOperator{\Bij}{Bij}

%--------- ENVIRONMENTS BEYOND CLASS FILE -----------------

\renewcommand{\thesection}

\newtheorem*{thm}{Theorem}
\newtheorem*{lem}{Lemma}
\newtheorem*{prop}{Proposition}
\newtheorem*{cor}{Corollary}

\theoremstyle{remark}
\newtheorem*{remark}{Remark}
\newtheorem*{notation}{Notation}

\theoremstyle{definition}
\newtheorem*{ex}{Example} 
\newtheorem*{defn}{Definition}

%-------------- BIBLIOGRAPHY STYLE / INDEX ---------------------

\bibliographystyle{plain}
\makeindex

\author{Daniel Cicala}
\title{My Math Dictionary}

\begin{document}

%--------------GOALS OF THIS-------------------
\section*{Goals}

The goals of this is to turn the category $\cat{F\t{-}Cospan(Sets)}$ as discussed in Brendan's paper \textit{Decorated Co-spans} into a bi-category using DPO Graph Rewriting.

%---------------SETUP --------------------
\section*{Setup}

As required in Brendan's construction, let $F \from (\cat{FinSet},+) \to (\cat{Set},\times)$ be a monoidal functor 
\begin{itemize}
	\item sending a set $n$ to the collection of graphs on nodes $n$
	\item sending a map $f \from x \to y$ of finite sets to the map $Fx \to Fy$ which assigns to each graph $e \xto{s,t} x$ the graph $e \xto{s,t} x \xto{f} y$.
\end{itemize}
Then an $F$-\define{decorated cospan} is a pair 
\[
	\( x \to n \unto y , \{\bullet\} \to Fn \)
\]
consisting of a co-span in $\cat{Sets}$ and a graph with node set $n$. A morphism of $f$-decorated co-spans 
\[
	(x \to n \unto y , \{\bullet\} \xto{s} Fn) \to (x \to m \unto y , \{\bullet\} \xto{t} Fm)
\]
is a set map $f \from n \to m$ such that the diagram
\[
	\begin{tikzcd}
		\{ \bullet \} \arrow[]{d}{s} \arrow[]{dr}{t} &
		{} \\
		Fn \arrow[]{r}{Ff}&
		Fm
	\end{tikzcd}
\]
commutes.

We hope to show that there is a bi-category $\cat{F\t{-}Cospan(Set)}$ with
\begin{enumerate}
	\item $0$-cells: finite sets,
	\item $1$-cells: iso-classes of $F$-decorate co-spans between sets, and
	\item $2$-cells: derivations $Fn_{\bullet} \Rightarrow Fm_{\bullet}$ via DPO graph transformations.
\end{enumerate}

Brendan has already shown that $\cat{F\t{-}Cospan(Set)}$ is a (hyper-graph) category with respect to the $0$-cells and $1$ cells.  Here, the composition of $1$-cells 
\[
	(x \to n \unto y , \{\bullet\} \xto{s} Fn ) ; (y \to m \unto z , \{\bullet\} \xto{t} Fm
)
\]
is the pair $(x \to n +_y m \unto y , \{ \bullet \} \to F(n +_y m))$ where 
\begin{itemize}
	\item $n +_y m$ is a pushout obtained from the diagram
\[
	\begin{tikzcd}[row sep = tiny,column sep = tiny]
		{} &
		{} &
		n +_y m	&
		{} &
		{} \\
		{} &
		n \arrow[]{ur}{ } &
		{} &
		m \arrow[]{ul}{ } &
		{} \\
		x	\arrow[]{ur}{ } &
		{} &
		y \arrow[]{ul}{j} \arrow[]{ur}{k}&
		{}&
		z \arrow[]{ul}{ }
	\end{tikzcd}
\]  
	\item the map $\{\bullet\} \to F(n +_y m)$ is the composite
\[
	\{\bullet\} \xto{\sigma^{-1}}
	\{\bullet\} \times \{\bullet\} \xto{(s,t)}
	Fn \times Fm \xto{\theta}
	F(n + m) \xto{F[j,k]}
	F(n +_y m) 
\]
where 
\begin{itemize}
	\item $\sigma$ is the unit from the monoidal structure (here it's the diagonal)
	\item $(s,t)$ is the product of the maps from the $F$-decorated cospans we are composing (here it picks out the corresponding graphs in $Fn$ and $Fm$,
	\item $\theta$ comes from the monoidalness of the functor $F$ (here is coproducts the chosen graphs in $Fn$ and $Fm$,,
	\item $F[j,k]$ is the image of the co-pairing of $j,k$ under $F$ (here it amalgamates the graph under $(y,j,k)$. 
\end{itemize}
\end{itemize}

%--------------DEFINITION OF BICATEGORY--------------

\section*{Bicategory Definition}

Recall that a \define{bi-category} $\cat{C}$ consists of the data:
\begin{itemize}
	\item collection of $0$-cells,
	\item for each pair $x,y$ of $0$-cells, a category $\cat{C} (x,y)$ whose objects, we call $1$-cells and morphisms we call $2$-cells,
	\item a functor $\op{id}_x \from \cat{1} \to \cat{C} (x,x)$ (i.e.\ a $1$-cell) that will represent an identity of for $1$-cells and $2$-cells which we will denote by $\cat{1}$ and use context to determine which identity we mean, and
	\item a functor $\circ_{x,y,z}$
\begin{align*}
	\cat{C} (x,y) \times \cat{C} (y,z) & \to \cat{C} (x,y) & \\
	(f,g) & \mapsto f;g & (\t{on } 1-\t{cells}) \\
	(\alpha, \beta) & \mapsto \alpha \ast \beta & (\t{on } 1-\t{cells})
\end{align*}
which is described by the diagram
\[
\begin{tikzcd}
x 
\ar[bend left=50]{r}[name=UL]{f}
\ar[bend right=50]{r}[name=BL,swap]{f'} &
y 	
\ar[bend left=50]{r}[name=UR]{g} 
\ar[bend right=50]{r}[name=BR,swap]{g'} &
z 
\ar[shorten <=3pt, shorten >=3pt,Rightarrow,to path={(UL) -- node[label=right:$\alpha$] {} (BL)}]{}
\ar[shorten <=3pt, shorten >=3pt,Rightarrow,to path={(UR) -- node[label=right:$\beta$] {} (BR)}]{}
\end{tikzcd}
\]
	\item Natural Isomorphisms 
\[
	\begin{tikzcd}[column sep=huge]
		\cat{C}(w,x) \times \cat{C} (x,y) \times \cat{C} (y,z) 
			\arrow[]{r}{\circ_{w,x,y} \times \cat{C} (y,z)} 
			\arrow[swap]{d}{\cat{C}(w,x) \times \circ_{x,y,z}} 
			&
		\cat{C} (w,y) \times \cat{C} (y,z) 
			\arrow[]{d}{\circ_{w,y,z}} 
			\\
		\cat{C} (w,x) \times \cat{C} (x,z) 
			\arrow[swap]{r}{\circ_{w,x,z}}
			\arrow[Rightarrow,shorten <=15pt, shorten >=15pt]{ur}{\alpha_{w,x,y,z}}
			&
		\cat{C} (w,z)
	\end{tikzcd}
\]
\[
	\begin{tikzcd}
		\cat{C}(x,y) \times \cat{1}
			\arrow[]{rd}{\cong}
			\arrow[swap]{d}{\cat{C}(x,y) \times id_y}
			&
		{}
			\\
		\cat{C}(x,y) \times \cat{C}(y,y)
			\arrow[swap]{r}{\circ_{x,y,y}}
			\arrow[Rightarrow, shorten <=3pt, shorten >=50pt]{ur}[near start]{r_{x,y}}
			&
		\cat{C} (x,y)
	\end{tikzcd}
	\begin{tikzcd}
		\cat{1} \times \cat{C}(x,y)
			\arrow[]{rd}{\cong}
			\arrow[swap]{d}{id_y \times \cat{C}(x,y)}
			&
		{}
			\\
		\cat{C}(x,x) \times \cat{C}(x,y)
			\arrow[swap]{r}{\circ_{x,x,y}}
			\arrow[Rightarrow, shorten <=3pt, shorten >=50pt]{ur}[near start]{\ell_{x,y}}
			&
		\cat{C} (x,y)
	\end{tikzcd}
\]
which are $2$-cells 
\begin{align*}
	& \alpha_{f,g,h} \from (f;g);h \xto{\cong} f;(g;h), \\ 
	& r_f \from \cat{1} ; f \xto{\cong} f, \t{and} \\
	& \ell_f \from f ; \cat{1} \xto{\cong} f
\end{align*}
	\item with the axioms that the following diagrams commute:
\[
	\begin{tikzcd}
		f;(g;(h;k)) 
			\arrow[]{rr}{1 \ast \alpha}
			\arrow[]{d}{\alpha}
			&
		{} 
			&
		f;((g;h);k) 
			\arrow[]{d}{\alpha}
			\\
		(f;g)(h;k) 
			\arrow[]{dr}{\alpha}
			&
		{} 
			&
		((f;g);h);k
			\arrow[]{dl}{\alpha \ast 1}
			\\
		{}
			&
		(f;(g;h));k
			&
		{}
	\end{tikzcd}
\]
\[
	\begin{tikzcd}
		f;(\t{id};g)
			\arrow[]{d}{\alpha}
			\arrow[]{dr}{\cat{1} \ast \ell}
			&
		{}
			\\
		(f;\t{id});g
			\arrow[]{r}{r \ast \cat{1}}
			&
		f;g		
	\end{tikzcd}
\]
\end{itemize}	

%-----------DPO GRAPH REWRITING--------

\section*{DPO Graph Rewriting}

Here we describe DPO graph rewriting.  Start with a \define{production}, or a span $L \unto K \to R$ of graphs where the left-hand morphism is an inclusion.  Given such a production, a graph morphism $L \to G$ is called a \define{matching morphism}.  Now, starting with a production and a compatible matching morphism, we can complete the diagram
\begin{equation} \label{diag:ProdNMatch}
	\begin{tikzcd}
		L \arrow[]{d}{g} &
		K \arrow[]{l}{} \arrow[]{r}{}&
		R \\
		G
	\end{tikzcd}
\end{equation}
to a double pushout
\begin{equation} \label{diag:DPO}
	\begin{tikzcd} 
		L \arrow[]{d}{} &
		K \arrow[]{l}{} \arrow[]{r}{} \arrow[]{d}{}&
		R \arrow[]{d}{} \\
		G  &
		D \arrow[]{l}{} \arrow[]{r}{} &
		G'
	\end{tikzcd}
\end{equation}
with the following process:
\begin{enumerate}
	\item Start with a compatible production and matching morphism as in \eqref{diag:ProdNMatch}.  Check
		\begin{itemize}
			\item \textit{(Dangling Condition)} that there is no edge in $G \sm g(L)$ incident to a node in $g(L) \sm g(K)$, and
			\item \textit{(Identification Condition)} that, given distinct nodes or edges $x,y \in L$, we have $g(x) = g(y)$ only if $x,y \in K$.
		\end{itemize}
	\item Construct $D$ by removing $g(L) \sm g(K)$ from $G$ and let $K \to D$ be the restriction of $g$ and $D \to G$ be the inclusion.
	\item Let $G'$ be the pushout of $D \unto K \to R$.
\end{enumerate}
Thus we have obtained \eqref{diag:DPO}. If we denote the span $L \unto K \to R$ by $p$, we say $G'$ is a \define{direct derivation} of $G$, writen $G \Rightarrow_p G'$, if the inside squares in \eqref{diag:DPO} are graph pushouts. Let $P$ be a collection of productions.  Then a graph $H$ is a \define{derivation} of $G$ is there are $p_1, \dotsc, p_k$ in $P$ and graphs $G_1, \dotsc, G_{k-1}$ such that $G \Rightarrow_{p_1} G_1 \Rightarrow_{p_2} \dotsm G_{k-1}\Rightarrow_{p_k} H$.  Denote this by $G \Rightarrow_P H$.

We will consider the tradition approach to DPO graph transformations which requires that the right-hand morphism $K \to R$ is injective and the matching morphism can be arbitrary.  




%----------CENTRAL IDEA-------------

\section*{Central Idea}

Given a pair of finite sets $x,y$, we want to form a category $\cat{FinSet}(x,y)$ whose objects are $F$-decorated co-spans and morphisms come from DPO graph transformations.  So, let's fix a set of productions $P = \{ L \to K \unto R \}$ and construct a graph  whose nodes are the $F$-decorated co-spans (which can be unambiguously denoted by $Fn_{k})$ which is a graph in the set $Fn$ of graphs with nodes $n$ and we use number $k$ to distinguish between different graphs on the same set $n$ of nodes) and edges $Fn_0 \to Fm_0$ whenever $Fn_0 \Rightarrow_P Fm_0$. Define $\cat{FinSet}(x,y)$ to be the free category on this graph.  

%----------THE DIRTY DETAILS-----------

\section*{The Dirty Details}

It remains to show that the axioms for a bicategory are satisfied.        











	
\end{document}